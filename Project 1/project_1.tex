\documentclass[12pt]{article}
\usepackage[utf8]{inputenc}
\usepackage{listings}
\usepackage{color}
\usepackage{hyperref} 
\usepackage[a4paper, total={16cm, 24cm}, top=2.5cm]{geometry}

\definecolor{mygreen}{rgb}{0,0.6,0}
\definecolor{mygray}{rgb}{0.5,0.5,0.5}
\definecolor{mymauve}{rgb}{0.58,0,0.82}

\lstset{language=C++,
    basicstyle=\fontsize{10}{13}\ttfamily,
    keywordstyle=\color{blue}\ttfamily,
    stringstyle=\color{red}\ttfamily,
    commentstyle=\color{green}\ttfamily,
    morecomment=[l][\color{magenta}]{\#},
    showstringspaces={false},
    tabsize=2
}


\title{CPP project 1}
\author{Jonas Nylund \and Anton Finnson}
\date{September 2020}

\begin{document}

\maketitle

The code for this lab can be found at \href{https://github.com/jonasnylund/CPP-project-1}{https://github.com/jonasnylund/CPP-project-1}

\section{Task 1}
We separately declared and defined the Taylor functions even though the program is very small, to see that we understood how to do it.

Neither of the two functions reassign their input variables \lstinline{N, x}, so to reduce the number of copied variables we set the types to \lstinline{const int&} and \lstinline{const double&} respectively.

As asked we verified that each error is bounded by the $N+1^{\rm{st}}$ term, for $N \in \{3,5,10\}$ and $x \in \{−1,1,2,3,5,10\}$. For all 18 choices of $(N,x)$ we printed whether the error was bounded by $N+1^{\rm{st}}$ term, both for $\sin$ and $\cos$. These results were counted, giving the following final output:

\lstinline{Bounded errors: 36}

\lstinline{Unbounded errors: 0}

\flushleft{For higher values of $N$ the limit in precision when calculating the error as well as the $N+1^{\rm{st}}$ term resulted in some errors being marked as unbounded.}

\subsection*{\texttt{trig.cpp}}
\lstinputlisting[language=C++]{1-1_trig/trig.cpp}

\subsection*{\texttt{taylor.hpp}}
\lstinputlisting[language=C++]{1-1_trig/taylor.hpp}

\subsection*{\texttt{taylor.cpp}}
\lstinputlisting[language=C++]{1-1_trig/taylor.cpp}

\newpage
\section{Task 2}
For this task we declared and defined the functions in the main file as the program is so small. We separated the files properly in task 1 to show that we know how to do it.

The algorithms was implemented as described, and tested for the tolerances $\texttt{tol} \in \{10^{-2}, 10^{-3}, 10^{-4}\}$. The value we used as the real value for the integral was calculated using WolframAlpha.
\begin{verbatim}
    Tolerance: 0.010000
    Result: 2.505996
    Error: 0.005187
    --------------
    Tolerance: 0.001000
    Result: 2.499857
    Error: 0.000953
    --------------
    Tolerance: 0.000100
    Result: 2.500810
    Error: 0.000001
    --------------
\end{verbatim}

The results seem very reasonable as they decrease with decreased tolerance. The errors are always smaller than the tolerance as well.

\subsection*{\texttt{adaptint.cpp}}
\lstinputlisting[language=C++]{1-2_adaptint/adaptint.cpp}
\end{document}
